It's an interesting world we live in where we are constantly on the move with our cellphones, laptops and other devices online day out and day in, all hours during the day.
With the increase of possibilities online we are eager to find a way to build trust online for even more feature rich online services for all of us as citizens.%, both in the private and public sector.

This thesis work was carried out at .SE (The Internet Infrastructure Foundation) in Sweden, at their Stockholm office in the spring of 2012, to be more precise, from the end of March to the beginning of June.
The subject for this thesis was chosen because of interest in internet security.
To be able to keep focus and dive deeper into the specific technologies Sophia's focus has been on identity federations and certificates while Christoffer has put more focus on DNS-Based Authentication of Named-Entites (DANE).

We would like to thank Staffan Hagnell (.SE) for giving us the opportunity to come and work for .SE and for the support on the way to our conclusions.
We would also like to thank Carl Ljungqvist (Certezza AB) for the first introduction about indentity federations and Rickard Bellgrim (Certezza AB) for giving us much needed help when setting up the testing environment as well as helping us understand how certificates are used within identity federations.
For giving us new perspectives about DANE we would like to thank Jakob Schlyter (Kirei) and Leif Johansson (SUNET). 

From our university, Lule\r{a} University of Technology, we would like to thank our 
supervisor Ulf Bodin, Dept. of Computer Science, Electrical and Space Engineering, 
for guiding us in the right direction when struggling with the thesis and putting words to our conclusions.
We would also like to thank Johan Carlson, Dept. of Computer Science, Electrical and Space Engineering, 
for the LaTeX template that we have used for this report. 

As a final note we would like to thank all of you that has supported us in our work but didn't get mentioned above.
Thank you.




% It's an interesting world we live in where we are constantly on the move with our cellphones, laptops and other devices online day out and day in, all hours during the day.
% with the increase of possibilities online we are eager to find a way to build trust 
% online for even more feature rich online services directed to you specifically as a citizen, both in the private and public sector.
%
% With the advant of DNSSEC shaping up in production around the world new possiblities arises.

% PREFACE Explains when, where, and why the work was carried out and you can also
% acknowledge the people who have supported and helped you during the course of your
% work.

% The preface is the place where you can discuss some personal
% experiences with the project. This is also the place where you can
% thank people who helped you out.

\vspace*{1.5cm}%
\hfill Sophia Bergendahl and Christoffer Holmstedt
