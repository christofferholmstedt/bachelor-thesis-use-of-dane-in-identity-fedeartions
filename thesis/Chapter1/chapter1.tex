\section{Background}
The topic of this thesis was chosen after a discussion with Staffan Hagnell(.SE), Jakob Schlyter(Kirei) and Carl Ljungqvist(Certezza).
Interesting topics are most of the time found at the leading edge of research, this was also true with this thesis. The discussion started out with trying to find a current problem with DNSSEC that would be suitable for a thesis.

\section{Problem description}

\section{Purpose}
The purpose with this theasis is to show proof of concept, that DANE (DNS-based Authentication of Named Entities) can be used to improve security aswell as the usage of DNSSEC (DNS Security Extenstions), since DANE is depending on the use of DNSSEC.

\section{Project delimitations}
In the sections above, DNSSEC is mentioned, there will be no detailed review in this report on how it works, the focus is on DANE and Identity Federations. 

---
INTRODUCTION 

Background: A short background, the reason why the work was carried out, and why you chose that particular topic.

Problem description: Problem area A broader discussion of what you are going to write about (the task). This can often be included under the same heading as the introduction. Formulating the problem leads to the purpose.

Purpose: The purpose may for example be specified in a list of to clauses.

Project delimitation: Arguments and boundaries. You must naturally give the reasons for your delimitations.
