\section{Background}
The topic of this thesis was chosen after discussion with Staffan Hagnell(.SE), Jakob Schlyter(Kirei) and Carl Ljungqvist(Certezza).
Interesting topics are most of the time found at the leading edge of research, this was also true with this thesis.
The discussion started out with trying to find a current problem with DNSSEC that would be suitable for a thesis.
In the end the topic on how to use DANE to improve security for identity federations was chosen.

\section{Problem description}
In todays society more and more information are shared through internet.
A lot of the communication consist of sharing pictures, stories and information with friends and family.
The online websites that allow and promote these services are often very easy to use and especially easy to register a new account for.
To login to most online websites the required credentials are often just username and password, which is deemed as a low security level.
Though in the end, the aboved mentioned services might not need higher security.

Services that require more security and especially the services that require somekind of proof that the user really is the person he/she says he/she is online, are lagging behind in the online era.
A solution to this problem exist in Sweden with "BankID" which was introduced in 2003\cite{website:bankid-about}.
The solution has its pros and cons and a new solution is in the horizon.
The new solution will be built upon the concept of identity federations.

As of today two major identity federations exists in Sweden.
Thoose are SWAMID for students in higher education run by SUNET and "Skolfederationen" for pupils/parents and teachers in compulsary "K-12" school system.

%Vi kopplar upp oss mer och mer 
%Delar bilder och händelser med varandra med bara några klick
%Detta görs till tjänster som använder enkelt lösenord/användarnamn där jag kan registrera mig med några knapptryck
%Det som släpar efter är tjänster där en koppling från en användare av en internettjänst kopplas till en fysisk eller juridisk personer är ett måste.
%Detta beror främst på tidiga lösningar på e-legitimationer har haft sina brister så som att inte vara tekniskt oberoende, dyra att implementera för tjänsteleverantörer/identitetsleverantörer och flera olika implementationer har gjorts vilket göra de olika varianterna inte är kompatibla med varandra.
%Lösningen på detta är på gång i Sverige med e-legitimationsnämnden (SOU) och e-legitimationer.
%Lösningen bygger på identitetsfederationer med SAML
%Problemet vi ställs inför är att ta reda på om vi kan och i sådana fall hur vi kan använda DANE för att förbättra säkerheten inom identitetsfederationer.


\section{Purpose}
The purpose with this theasis is to show proof of concept, that DANE (DNS-based Authentication of Named Entities) can be used to improve security aswell as the usage of DNSSEC (DNS Security Extenstions), since DANE is depending on the use of DNSSEC.

\section{Project delimitations}
In the sections above, DNSSEC is mentioned, there will be no detailed review in this report on how it works, the focus is on DANE and Identity Federations. 

---
INTRODUCTION 

Background: A short background, the reason why the work was carried out, and why you chose that particular topic.

Problem description: Problem area A broader discussion of what you are going to write about (the task). This can often be included under the same heading as the introduction. Formulating the problem leads to the purpose.

Purpose: The purpose may for example be specified in a list of to clauses.

Project delimitation: Arguments and boundaries. You must naturally give the reasons for your delimitations.
