This document serves both as an example of the \LaTeX\ template
for Master's theses, and at the same time as documentation
on how to use it.

\section{General principles}
Use a main document to set up title page and chapters. In this way
you can compile a subset of the entire thesis by simply commenting
the \texttt{$\backslash$input\{...\}} commands in the main document.
Typeset the document by compiling the main document. The
subdocuments can not be compiled by themselves.

The subdocuments should not contain any document style
information. Just start writing the contents as in this file
(\texttt{Chapter1/chapter1.tex})

\section{Revision history}
The list below shows changes made to the template for each version
and what (if any) changes you will need to do in your main
document after upgrading the .cls file. The main document of this
example will always be written for the current release of the
template.
%
\begin{itemize}
    \item Version 0.9: First version.
    \item Version 1.0, released November 17, 2004.
    \begin{itemize}
        \item Various bug fixes (e.g.\ running headers,
        float-to-text spacing)
        \item Changed typeface for body text to \textit{Computer
        Modern}. Heading font still \textit{Helvetica}.
        \item Increased line spacing (5 \%) for improved
        readability.
        \item New language option for the document class. You can
        now select Swedish (sv) or English (en). See main document
        for changes. Note that the
        \texttt{$\backslash$makepreamble} command now requires an
        additional argument, containing a Swedish abstract. If you
        write the report in English, just leave this argument
        empty. For Swedish reports, the English abstract is still
        mandatory.
        \item Added support for onesided or twosided print. See
        main document for class declaration options.
    \end{itemize}
    \item Version 1.01, released November 19, 2004.
    \begin{itemize}
        \item Fixed bug in formatting of Bibliography list
        heading.
        \item Added possibility to change the paragraph
        separation (and updated the .cls to manage this. See main
        document for more information.
        \item Two more files (swedish.ldf and english.ldf) are now needed and should be placed
        in the same directory as the document. They contain
        updates for the ``babel" package used to manage multiple
        languages.
    \end{itemize}
    \item Version 1.02, released February 22, 2005.
    \begin{itemize}
        \item Fixed a bug in the bibliography list (occurs if the list is more than one page). There is no
        change to the document class, but to the main document. See
        the main document of this example.
    \end{itemize}
    \item Version 1.03, released March 17, 2005.
    \begin{itemize}
        \item Fixed minor page numbering bug.
    \end{itemize}
    \item Version 1.04, released March 17, 2005.
    \begin{itemize}
        \item Fixed row spacing (paragraph breaking) class option
        (see this document for info).
        \item Fixed language selection. Only class header
        declaration necessary now. Previous versions required a
        separate \texttt{$\backslash$selectlanguage} command.
    \end{itemize}
    \item Version 1.05, released March 24, 2005.
    \begin{itemize}
        \item Fixed the \texttt{$\backslash$cleardoublepage} command so that cleared
        pages are completely empty. Thanks David!
        \item Fixed figure and table numbering issue in appendices.
        \item Fixed a bug that caused page numbers to mysteriously
        disappear. 
    \end{itemize}
	 \item Version 1.06, released November 20, 2008.
	 \begin{itemize}
		  \item Fixed a bug regarding \texttt{$\backslash$belowcaptionskip} and 
			the \texttt{listings} package. Thanks to the guys at the latex-community forum!
	 \end{itemize}
	 \item Version 1.5, released June 7, 2009.
    \begin{itemize}
        \item Various minor bug fixes (hardly anyone even noticed). 
        \item Name change! The template is now called \texttt{csee\_msc\_thesis}
	 \end{itemize}
	 \item Version 1.6, released October 28, 2009.
	 \begin{itemize}
			\item Table of contents heading is now right-justified, just like all other chapter headings.
	  \end{itemize}
	 \item Version 1.7, released November 27, 2009.
	 \begin{itemize}
	    \item Fixed a bug in the numbering of tables.
		 \item Increased the number of section levels included in the table of contents.
	 \end{itemize}
\end{itemize}

\section{Making the document preamble}
The document preamble is a series of pages containing:
\begin{itemize}
    \item A title page,
    \item an abstract (or two if you use the \textit{Swedish} class option,
    \item and a preface page.
\end{itemize}

The title, author, address, abstract, preface are defined in the
main document. To actually include it into the thesis, use the
\texttt{$\backslash$startpreamble} command provided by the
document class. See the code of the main document for an example
of how to use this. In this example, the abstract and preface are
stored in separate files (containing only the actual text).

\section{Writing the rest of the report}
To add chapters, simply include one more in the main document. The
chapter headings are automatically generated by the document
template. See Chapter \ref{ch2} on how to include graphics.

\section{Including appendices}
At the end of the thesis, after all chapters but before the
reference list, you might want to add some appendices. To do this,
simply type \texttt{$\backslash$appendix} before the
\texttt{$\backslash$thebibliography} command and then add each
appendix with the \texttt{$\backslash$makeappendix} command
(identical syntax as the \texttt{$\backslash$makechapter}
command). See Appendix \ref{appA} for a completely meaningless
mathematical derivation.

\section{Including bibliography references}
At the end of the document you will have the list of bibliography
references. In this thesis the \textsc{BibTeX} reference system is
used. Add your references to a database file and the reference
list will be generated automatically. For more details on how to
use the \textsc{BibTeX} system, please refer to any good \LaTeX\
documentation, e.g.\ \cite{Lamport} or \cite{Goossens}.\nocite{*}
