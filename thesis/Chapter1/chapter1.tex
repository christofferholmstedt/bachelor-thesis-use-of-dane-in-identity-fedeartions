\section{Background}
The topic of this thesis was chosen after discussion with Staffan Hagnell(.SE), Jakob Schlyter(Kirei), Rickard Bellgrim (Certezza AB) and Carl Ljungqvist(Certezza AB).
Interesting topics are most of the time found at the leading edge of research, this was also true with this thesis.
The discussion started out with trying to find a current problem with DNSSEC that would be suitable for a thesis.
In the end, the topic on how to use DANE to improve security for identity federations was chosen.

\section{Problem description}
In todays society more and more information is shared through internet.
A lot of the communication consist of sharing pictures, stories and information with friends and family.
The online websites that allow and promote these services are often very easy to use and especially easy to register a new account for.
To login to most of these services the required credentials are often username and password, which is deemed as a low assurance level\cite[p.~244]{pdf:SOU}.
This has its pros and cons, though in the end, the above mentioned services might not need higher assurance level.

Services that require somekind of proof that the user behind the keyboard really is the person he/she says he/she is, are lagging behind in the online era.
A solution to this problem is "BankID" which was introduced in 2003\cite{website:bankid-about} in Sweden.
Other solutions comes from Nordea, SEB, Telia, Posten and Steria\cite[p.~256]{pdf:SOU}.
All listed solutions have their pros and cons and a new solution is in the horizon.
The new solution will be built upon the concept of identity federations\cite[p.~23]{pdf:SOU}.

As of today two major identity federations exists in Sweden.
Thoose are SWAMID for students in higher education run by SUNET and "Skolfederationen" for pupils/parents and teachers in compulsary primary and secondary school.

\section{Purpose}
The purpose with this theasis is to show proof of concept, that DANE (DNS-based Authentication of Named Entities) can be used to improve security aswell as the usage of DNSSEC (DNS Security Extenstions), since DANE is depending on the use of DNSSEC.

\section{Project delimitations}
In the sections above, DNSSEC is mentioned, there will be no detailed review in this report on how it works, the focus is on DANE and Identity Federations. 

---
INTRODUCTION 

Background: A short background, the reason why the work was carried out, and why you chose that particular topic.

Problem description: Problem area A broader discussion of what you are going to write about (the task). This can often be included under the same heading as the introduction. Formulating the problem leads to the purpose.

Purpose: The purpose may for example be specified in a list of to clauses.

Project delimitation: Arguments and boundaries. You must naturally give the reasons for your delimitations.
