\section{How to improve security for Identity federations}
\section{How to combine Identity federations and DANE}
As stated earlier it's possible to divide federations into two categories, those with a federation provider and those without.
In the analysis that has been done the solution with a federation provider is the most probable and viable setup in the long run, it's basiclly much easier to scale, both technically and non-technically.
It's with this configuration in mind the following evaluation has been made.

%TODO: (1) Kontrollera att det finns information om båda kategorierna av federationer.
%TODO: (2) Kontrollera att analys är gjord och att det verkligen är varianten med en FO som är mest trolig.


% \cite[p.~256]{pdf:SOU}

\section{Temporary section}
EVALUATION This is where you put the spotlight on your solution's strengths and
weaknesses. An evaluation must be made of the methods/theory applied and the validity
and reliability of the data. You should describe the most interesting results of your work in a
results section. Remember that less important but complementary results can to advantage
be placed in appendices. A rule of thumb is that all results that you use in your analysis and
in your conclusions should be reported in the Evaluation chapter and the rest reported in
appendices.

