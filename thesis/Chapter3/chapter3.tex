\section{Identity federations}

\section{Certificates} 

\section{DANE} 
DANE, DNS-based Authentication of Named Entities, is as of writing still a new concept and technology.
It's introduced with the informational RFC document\cite{rfc:6394} describing some use cases and specified in more detail in the Internet-Draft "The DNS-Based Authentication of Named Entities (DANE) Protocol for Transport Layer Security (TLS)"\cite{rfc:draft-dane}.
Make note that the Internet-Draft from the Internet Engineering Task Force (IETF) is still "work in progress" as it's still a draft.





%Technical view, what do we have?, how do they work one by one? DNS, DNSSEC, DANE, Identity Federations (SP, IDP)...?

%This investigation 

%DNS (Domain Name System) is the Internets domain name system, a hierarchic and distributed database that makes it possible to quickly find information about which IP-address (and other information) that is connected to a domain name. \cite[p.~64]{book:guide_dns}

%DNSSEC (Domain Name Security Extensions) is a securitylayer added to DNS to secure system against assults like cachingpoisning through signing DNS-data digitaly.\cite[p.~64]{book:guide_dns}



%\section{Theories}

%How do they work together?

%\section{Tools}

%Software? Shibboleth?


%THEORY 
%Background, 
%theories, 
%tools, 
%and anything else that can be used in your work.
%These are largely determined by your purpose and delimitations. This puts your investigation
%in its scientific context.
