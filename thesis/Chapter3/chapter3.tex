\section{Approch}

The approach was to first read up about Identity Federations, SAML 2.0, Shibboleth, DNSSEC and DANE. This continued with setting up a test environment using Shiboleth to start analyzing how the Identity Federation works and if or where DANE can be implemented. The analysis provided information about possible places and the implementation started.  

\section{Investigation description}


\section{Discussion of alternative methods}


\section{Reliability and validity}


Joppe (2000) defines reliability as:


...The extent to which results are consistent over time and an accurate representation of
the total population under study is referred to as reliability and if the results of a study
can be reproduced under a similar methodology, then the research instrument is
considered to be reliable. (p. 1)




Joppe (2000) provides the following explanation of what validity is in quantitative
research:


Validity determines whether the research truly measures that which it was intended to
measure or how truthful the research results are. In other words, does the research
instrument allow you to hit 'the bull's eye' of your research object? Researchers
generally determine validity by asking a series of questions, and will often look for the
answers in the research of others. (p. 1)



-
-
-
-
-
METHOD in longer theses, it is appropriate for the author to define; 
-
-
- his or her approach.
-
-
- 
-All reports must contain a sufficiently detailed description of; 
-
-
- how the investigation was carried out as to allow it to be repeated by someone else and 
-
-
- a discussion of alternative methods(strengths and weaknesses). 
-
-
-
-
-The concepts; 
-
-
- reliability and validity have a given place when issues of method are dealt with.




