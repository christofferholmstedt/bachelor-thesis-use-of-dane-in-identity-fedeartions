To add a second chapter, simply include it in the main document.

\section{Including graphics}
The template requires the \texttt{$\backslash$graphicx} package
for figure import. The package is included by the template by
default.

To include graphics use, use the following code:\\[\baselineskip]
\shadowbox{%
\begin{minipage}[l]{0.85\textwidth}
\begin{verbatim}
  \begin{figure}[ht]
     \begin{center}
       \includegraphics[scale=1]{figures/ch2_fig1.pdf}
     \end{center}
     \caption{Example figure.\label{ch2:fig1}}
  \end{figure}
\end{verbatim}
\end{minipage}
}\\[\baselineskip]

This example will compile with \texttt{pdfLaTeX} and import the
figure \texttt{ch2\_fig1.pdf} stored in the \texttt{Figures}
directory. If you wish to compile with \texttt{latex} instead of
\texttt{pdflatex}, all pdf-figures have to be available as EPS
(encapsulated PostScript) instead.

\section{Special features}
For some reason (currently unknown...) there has been a problem
writing bold-face greek letters. A quick and dirty fix to the
problem is included in the document class for some greek letters.
See the .cls file for complete list.

Example: to define a bold $\theta$, the template defines the name
\texttt{$\backslash$bftheta}, which is declared as
\texttt{$\backslash$def$\backslash$bftheta\{$\backslash$mbox\{$\backslash$boldmath
$\backslash$theta\}\}}. This will be printed as $\bftheta$.
