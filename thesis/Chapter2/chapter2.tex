\section{Approach}

The approach was to first read up about the subject, identity federations as well as talk to experts. 
Identity federations is based on the SAML 2.0 standard, other standards as well,
but we chose to read up about SAML because it is what is used in both SWAMID and skolfederation.se. 

It was due to recommendations the software choice to use Shibboleth was made, 
instead of simpleSAMLphp or equivalent software.
 
To understand identity federations, SAML and Shibboleth better, we set up a test environment using Shibboleth. 
In the test environment only one service provider and one identity provider is implemented, 
this choice does not affect the analysis result in our opinion. 
More about the service- and identity provider and other entities in the identity federation is described in the thesis. 

After reading up on identity federations, we read up about DANE, through drafts, by following the DANE mailing list and by talking to experts. DANE demands that DNSSEC is used and therefore some reading was also done about DNSSEC. We also read up more about TLS/SSL to easier understand where DANE is needed in the identity federation.

When the analysis was finished, we had found possible places where DANE could be used and we could start implementing.

%\section{Investigation description}

%\section{Discussion of alternative methods}

%\section{Reliability and validity}


%METHOD in longer theses, it is appropriate for the author to define; 
% his or her approach.
% All reports must contain a sufficiently detailed description of; 
% how the investigation was carried out as to allow it to be repeated by someone else and 
% a discussion of alternative methods(strengths and weaknesses). 
% The concepts; 
% reliability and validity have a given place when issues of method are dealt with.




