For this project git together with github has been used for revision control.
To be able to keep the installation and configuration instructions up to date even after this report has gone into print we have decided to write a summary of all used software in this appendix and then put all concrete instructions on github.

As git is a distributed version control system there is no specific repository that is our "master" that we can link to.
Due to this both of our repositories could be used but to not flood this appendix with redundent/"duplicate" links all links in this appendix will be to the repository available under Christoffer's github account.

This report in LaTeX as well as all of our instructions are available at \\
\url{https://github.com/christofferholmstedt/BachelorThesis}

The DANE Trust Engine Extension is available at \\
\url{https://github.com/christofferholmstedt/shibIdPextension}

\section{Bind, OpenDNSSEC and other requirements}
To be able to use DANE, DNSSEC needs to be up and running as a prerequisites.
In this project Bind as DNS server together with OpenDNSSEC to sign the zone were used.

To sign the zone you need some kind of key store from where OpenDNSSEC can fetch the keys.
If a Hardware Security Module is not available SoftHSM can be used for this, also available from the OpenDNSSEC website.
(\url{http://www.opendnssec.org/})

For user authentication on the identity provider server we have used OpenLDAP \cite{website:openldap}.
Complete installation and configuration instructions are available at our github repository.

\section{Shibboleth Service/Identity Provider}
We installed the service- and identity provider software on different virtual machines to better be able to mimic a production setup.
It should be possible to install them on the same machine but as the service provider use Apache HTTP Server and the identity provider use Apache Tomcat some complications may arise concerning the configuration on which webserver should listen on which ports.

Complete installation and configuration instructions are available at our github repository.

\section{Federation Operator - Metadata Aggregator}
One of the Federation Operator technical purposes is to collect, combine and share the metadata from and to respective service/identity provider.
For our testing environment we downloaded the metadata from respective provider to the federation operator.
We then got help from Leif Johansson (SUNET) who provided us with a link to the scripts that is used to sign the metadata for identity federation "Skolfederationen".

More information is available at the github repository mentioned above.

\section{DANE Trust Engine Extension}
The code for the DANE Trust Engine Extension is available at github \url{https://github.com/christofferholmstedt/shibIdPextension}.
As of today the code in the master branch compile just fine and Shibboleth loads the extension but it's not possible to activate the extension by configuring it in the 'relying-party.xml' file with Shibboleth.
The current code is a copy of the "Static Explicit Key Signature Trust Engine" which is built in with the standard distribution of Shibboleth.
Some outcommented code is for doing a DNS lookup tested in a seperate code project and should work when our extension is able to start properly.

Instructions on how to install, configure and continue the development on the extension is available at the git repository. 

%\section{TEMP Section}
%TODO Here is a "nocite" latex command that needs to be removed when final printing is to be done. The nocite command below include all references in our bibliography but it's not certain if we have used them all.
%This text finishes with a reference to our fake reference.
%In the references list all references above this one is used in our text and all others are orphans to be removed if we don't need them later on.\cite{fake}
%\nocite{*}
