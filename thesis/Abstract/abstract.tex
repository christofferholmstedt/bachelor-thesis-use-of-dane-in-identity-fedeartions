Identity federations is a way to build a standard for online services within the 
federation similar to the one in real life with identification cards and signatures.
However, there are more security aspects to take into account online, this report 
analysis the security mechanism used to achieve data integrity in an identity 
federation and specifically the use of X.509 certificates as well as an evaluation of the 
possibility to use DANE to improve the security for an identity federation.
Moreover, the report describes different protocols, standards and a lot about the 
underlying technology that is used in identity federations and DANE. 
The report is a result of litterateur studies, set up of test environment and discussions 
with experts and between us. It concludes that DANE can be used for improvements, 
that SAML 2.0 that our identity federation is based on, its documents needs to more 
restricted and the metadata sharing within SAML is the achilles heel.

%need trust in identity federations
%therefore increasing security is important
%this is what this is about, how to use DANE with identity federations.

%ABSTRACT \- what the report is about and puts it in a global perspective.

%Abstract to be written TODO

%This is an example document of how to use the
%\texttt{csee\_msc\_thesis} document class.

%The document class supports both Swedish and English theses, double
%or single sided print.
