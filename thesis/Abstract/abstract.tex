Identity federations is a way to build a standard for online services similar to the one in real life with identification cards and signatures.
However, there are more security aspects to take into account online, this report analysis the security mechanism used to achieve data integrity in an identity federation and specifically the use of X.509 certificates as well as an evaluation of the possibility to use DNS-based Authentication of Named Entities (DANE) to improve the security for an identity federation.
Moreover, the report describes different protocols, standards and some about the underlying technology that is used in identity federations and DANE. 
The report is a result of litterateur studies, set up of test environment, discussions with experts and between ourselves.
It concludes that improvements can be made on how identity federations handle their own metadata, trust other entities metadata and that DANE can be used to improve the initial trust bonding.

%and its documents needs to more restricted and the metadata sharing within SAML is the achilles heel.
%that DANE can be used for improvements
%need trust in identity federations
%therefore increasing security is important
%this is what this is about, how to use DANE with identity federations.

%ABSTRACT \- what the report is about and puts it in a global perspective.

%Abstract to be written TODO

%This is an example document of how to use the
%\texttt{csee\_msc\_thesis} document class.

%The document class supports both Swedish and English theses, double
%or single sided print.
